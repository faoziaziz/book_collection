% Created 2023-01-12 Thu 15:53
% Intended LaTeX compiler: pdflatex
\documentclass[11pt]{article}
\usepackage[utf8]{inputenc}
\usepackage[T1]{fontenc}
\usepackage{graphicx}
\usepackage{longtable}
\usepackage{wrapfig}
\usepackage{rotating}
\usepackage[normalem]{ulem}
\usepackage{amsmath}
\usepackage{amssymb}
\usepackage{capt-of}
\usepackage{hyperref}
\author{Aziz Faozi}
\date{\today}
\title{Tutorial Emacs}
\hypersetup{
 pdfauthor={Aziz Faozi},
 pdftitle={Tutorial Emacs},
 pdfkeywords={},
 pdfsubject={},
 pdfcreator={Emacs 28.1 (Org mode 9.5.2)}, 
 pdflang={English}}
\begin{document}

\maketitle
\tableofcontents



\section{Pengantar}
\label{sec:org24b4503}
Ini sekadar tutorial untuk emacs

\subsection{HotKey dalam emacs}
\label{sec:org6798ba4}
\begin{center}
\begin{tabular}{ll}
command & deskripsi\\
C - x C - f & untuk membuka file\\
C - x C - s & untuk mengesave file\\
C - x k & untuk kill screen\\
C - x C - c & menutup emacs\\
C - x 2 & Split verticals\\
C - x 3 & Split horizontal\\
C - x o & Pindah screen\\
C - x 1 & Untuk memfokuskan screen\\
\end{tabular}
\end{center}




\section{Notes}
\label{sec:org2307bfc}
Untuk menggenerate PDF anda harus menginstall pdflatex. Jika anda menggunakan windows anda bisa menginstallnya di
\href{https://mirror.ctan.org/systems/texlive/tlnet/install-tl-windows.exe}{Link TextLive}, kalian kemudian bisa publish dengan command C-c C-e l p
\end{document}