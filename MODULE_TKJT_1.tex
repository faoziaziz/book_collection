% Created 2023-01-12 Thu 15:40
% Intended LaTeX compiler: pdflatex
\documentclass[11pt]{article}
\usepackage[utf8]{inputenc}
\usepackage[T1]{fontenc}
\usepackage{graphicx}
\usepackage{longtable}
\usepackage{wrapfig}
\usepackage{rotating}
\usepackage[normalem]{ulem}
\usepackage{amsmath}
\usepackage{amssymb}
\usepackage{capt-of}
\usepackage{hyperref}
\author{Aziz Amerul Faozi}
\date{\today}
\title{MODULE E Teknik Jaringan Komputer Dan Telekomunikasi}
\hypersetup{
 pdfauthor={Aziz Amerul Faozi},
 pdftitle={MODULE E Teknik Jaringan Komputer Dan Telekomunikasi},
 pdfkeywords={},
 pdfsubject={},
 pdfcreator={Emacs 28.1 (Org mode 9.5.2)}, 
 pdflang={English}}
\begin{document}

\maketitle
\tableofcontents


\section{Pengantar}
\label{sec:org48dd56d}
Modul ini dibuat untuk sebagai materi yang diajarkan pada kelas TJKT di SMK.

\subsection{Desain Kurikulum}
\label{sec:org67fb6ec}
\subsubsection{Memahami proses bisnis pada bidang teknik komputer dan telekomunikasi.}
\label{sec:org29f875b}
\begin{enumerate}
\item constumer handling
\label{sec:org9b259ed}
\item perencanaan
\label{sec:org4c0d9bb}
\item analisis kebutuhan pelanggan
\label{sec:orgdebf4ee}
\item strategi implementasi (instalasi, konfigurasi, monitoring)
\label{sec:org7009570}
\item pelayanan pada pelanggan sebagai implementasi penerapan budaya mutu
\label{sec:org91daec3}
\end{enumerate}
\subsubsection{Memahami perkembangan teknologi pada perangkat teknik jaringan komputer dan telekomunikasi}
\label{sec:org31bdac6}
\begin{enumerate}
\item Pengetahuan tentang 5G, Microware Link, IPV6, teknologi serat optik terkini.
\label{sec:org71297d1}
\item Pengetahuan IoT, Data Center, Cloud Computing dan Information Security
\label{sec:org721091a}
\item Pengetahuan tengan isu-isu implementasi teknologi jaringan dan telekomunikasi antara lain keamanan informasi, penetrasi internet.
\label{sec:orgbc8bc37}
\end{enumerate}
\subsubsection{Profesi dan Kewirausahaan (job-profile  dan technopreneur) di bidang teknik jaringan komputer dan telekomunikasi}
\label{sec:orgfcb0623}
\begin{enumerate}
\item Memahami jenis-jenis profesi kewirausahaan (job-profile dan technopreneurship, personal branding serta peluang usaha di bidang Teknik Jaringan Komputer dan Telekomunikasi)
\label{sec:orgbaa0241}
\item membangun vision dan passion dengan melaksanakan pembelajaran berbasis proyek nyata sebagai simulasi proyek kewirausahaan.
\label{sec:org6a9a35e}
\end{enumerate}

\subsection{Pengantar}
\label{sec:orgc9b8945}
\subsubsection{Membangun sebuah perangakat TJKT}
\label{sec:orgcd45f1e}
\begin{enumerate}
\item Membuat sebuah perangat radio
\label{sec:org94fc9d1}
\begin{enumerate}
\item Bahan
\label{sec:org89c58c4}
\begin{center}
\begin{tabular}{rlr}
no & Nama & Kuantitas\\
--- & ---- & -\\
1 & Resistor & 1\\
2 & Capacitor & 2\\
3 & inductor & 3\\
4 & transistor & 1\\
-- & - & -\\
\end{tabular}
\end{center}
\end{enumerate}
\end{enumerate}
\end{document}