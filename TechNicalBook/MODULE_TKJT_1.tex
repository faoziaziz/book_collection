% Created 2023-01-16 Mon 22:26
% Intended LaTeX compiler: pdflatex
\documentclass[11pt]{article}
\usepackage[utf8]{inputenc}
\usepackage[T1]{fontenc}
\usepackage{graphicx}
\usepackage{longtable}
\usepackage{wrapfig}
\usepackage{rotating}
\usepackage[normalem]{ulem}
\usepackage{amsmath}
\usepackage{amssymb}
\usepackage{capt-of}
\usepackage{hyperref}
\author{Aziz Amerul Faozi}
\date{\today}
\title{MODULE E Teknik Jaringan Komputer Dan Telekomunikasi}
\hypersetup{
 pdfauthor={Aziz Amerul Faozi},
 pdftitle={MODULE E Teknik Jaringan Komputer Dan Telekomunikasi},
 pdfkeywords={},
 pdfsubject={},
 pdfcreator={Emacs 28.1 (Org mode 9.5.2)}, 
 pdflang={English}}
\begin{document}

\maketitle
\tableofcontents


\section{Pengantar}
\label{sec:org46a0322}
Modul ini dibuat untuk sebagai materi yang diajarkan pada kelas TJKT di SMK.

\subsection{Desain Kurikulum}
\label{sec:org07904ba}
\subsubsection{Memahami proses bisnis pada bidang teknik komputer dan telekomunikasi.}
\label{sec:orgf4a38d0}
\begin{enumerate}
\item constumer handling
\label{sec:orgc7b6e85}
\item perencanaan
\label{sec:orga64a4d5}
\item analisis kebutuhan pelanggan
\label{sec:orgba6e038}
\item strategi implementasi (instalasi, konfigurasi, monitoring)
\label{sec:org3fb224e}
\item pelayanan pada pelanggan sebagai implementasi penerapan budaya mutu
\label{sec:orgd4635e4}
\end{enumerate}
\subsubsection{Memahami perkembangan teknologi pada perangkat teknik jaringan komputer dan telekomunikasi}
\label{sec:org7910125}
\begin{enumerate}
\item Pengetahuan tentang 5G, Microware Link, IPV6, teknologi serat optik terkini.
\label{sec:orgeb7500b}
\item Pengetahuan IoT, Data Center, Cloud Computing dan Information Security
\label{sec:orgfcc5688}
\item Pengetahuan tengan isu-isu implementasi teknologi jaringan dan telekomunikasi antara lain keamanan informasi, penetrasi internet.
\label{sec:orgb903a41}
\end{enumerate}
\subsubsection{Profesi dan Kewirausahaan (job-profile  dan technopreneur) di bidang teknik jaringan komputer dan telekomunikasi}
\label{sec:org5b81559}
\begin{enumerate}
\item Memahami jenis-jenis profesi kewirausahaan (job-profile dan technopreneurship, personal branding serta peluang usaha di bidang Teknik Jaringan Komputer dan Telekomunikasi)
\label{sec:org9485f55}
\item membangun vision dan passion dengan melaksanakan pembelajaran berbasis proyek nyata sebagai simulasi proyek kewirausahaan.
\label{sec:org4a98150}
\end{enumerate}
\subsubsection{Keselamatan dan Kesehatan Kerja Lingkungan Hidup (K3LH) dan budaya kerja industri}
\label{sec:org6d143c4}
\begin{enumerate}
\item Pada akhir fase E peserta didik mampu menerapkan K3LH dan budaya kerja industri, antara lain: praktik- praktik kerja yang aman, bahaya-bahaya di tempat kerja, prosedur- prosedur dalam keadaan darurat, dan penerapan budaya kerja industri (Ringkas, Rapi, Resik, Rawat, Rajin), termasuk pencegahan kecelakaan kerja di tempat tinggi dan prosedur kerja di tempat tinggi (pemanjatan).
\label{sec:orgb2f42f2}
\end{enumerate}

\subsubsection{Dasar-dasar teknik jaringan komputer dan telekomunikasi}
\label{sec:orgf5aa857}
\begin{enumerate}
\item Pada akhir fase E peserta didik mampu memahami tentang jenis alat ukur dan penggunaannya dalam pemeliharaan jaringan komputer dan sistem telekomunikasi.
\label{sec:org2921244}
\end{enumerate}
\subsubsection{Media dan jaringan telekomunikasi}
\label{sec:org40cb8b0}
\begin{enumerate}
\item Pada akhir fase E peserta didik mampu memahami prinsip dasar sistem IPV4/IPV6, TCP IP, Networking Service, Sistem Keamanan Jaringan Telekomunikasi, Sistem Seluler, Sistem Microwave, Sistem VSAT IP, Sistem Optik, dan Sistem WLAN.
\label{sec:orgc2a9326}
\end{enumerate}
\subsubsection{Penggunaan Alat Ukur}
\label{sec:org1e4f909}
\begin{enumerate}
\item Pada akhir fase E peserta didik mampu menggunakan alat ukur, termasuk pemeliharaan alat ukur untuk seluruh jaringan komputer dan sistem telekomunikasi.
\label{sec:org6767ae3}
\end{enumerate}

\subsection{Pengantar}
\label{sec:orga298139}
\subsubsection{Membangun sebuah perangakat TJKT}
\label{sec:org273a046}
\begin{enumerate}
\item Membuat sebuah perangat radio
\label{sec:org90e9f22}
\begin{enumerate}
\item Bahan
\label{sec:orgc9296fb}
\begin{center}
\begin{tabular}{rlr}
no & Nama & Kuantitas\\
--- & ---- & -\\
1 & Resistor & 1\\
2 & Capacitor & 2\\
3 & inductor & 3\\
4 & transistor & 1\\
-- & - & -\\
\end{tabular}
\end{center}
\end{enumerate}
\end{enumerate}
\end{document}